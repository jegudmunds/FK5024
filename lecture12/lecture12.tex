% !TEX encoding = UTF-8 Unicode
% !TEX TS-program = pdflatexmk

\documentclass[a4paper,12pt]{article}

\usepackage[utf8]{inputenc}
\usepackage{geometry}
\usepackage{redefine-sections}
\usepackage{amsmath}
\usepackage{amsthm}
\usepackage{graphicx}
\usepackage{fancyhdr}
\usepackage{tikz}
\usepackage{pstricks}
\usepackage{pst-node}
\usepackage{wrapfig}
\usepackage{graphicx}
\usepackage{bibspacing}
\usepackage{multicol}
\usepackage{csquotes}
\usepackage[numbers,sort&compress]{natbib}
\usepackage{hyperref}
\usepackage{wrapfig}
\usepackage{bm}

\makeatletter

% Redefine maketitle
%
\def\maketitle{%
\par\textbf{\@title}%
\par{\@author}%
\par}

% Redefine \em and \emph
%
\DeclareRobustCommand{\em}{%
  \@nomath\em \if b\expandafter\@car\f@series\@nil
  \normalfont \else \bfseries \fi}

\makeatother

\geometry{left=2cm,right=2cm,top=2.5cm,bottom=2.5cm}
\lhead{\textsc{FK5024}}
\rhead{\textsc{Lecture 12}}
\pagestyle{fancy}

\theoremstyle{remark}
\newtheorem*{example}{Example}
\setlength{\parindent}{0pt}
\setlength{\parskip}{1.5em}
\renewcommand{\familydefault}{\sfdefault}

%  frequent science terms
\newcommand{\lcdm}{$\mathrm{\Lambda CDM}$ }
\newcommand{\lcdmn}{$\mathrm{\Lambda CDM}$}
\newcommand{\etal}{et al.\@ }
\newcommand{\etaln}{et al.\@}
\newcommand{\mrm}[1]{\mathrm{#1}}

%% From Header.tex, see tmp/
%% http://www.dfcd.net/articles/latex/latex.html
\renewcommand{\v}[1]{\ensuremath{\mathbf{#1}}} % for vectors
\newcommand{\vv}[1]{\ensuremath{\vec{\mathbf{#1}}}} % for vectors2
\newcommand{\vvv}[1]{\ensuremath{{\bm{#1}}}} % for vectors2
\newcommand{\gv}[1]{\ensuremath{\mbox{\boldmath$ #1 $}}}
\newcommand{\ellp}{\ell '}
\newcommand{\uv}[1]{\ensuremath{\mathbf{\hat{#1}}}} % for unit vector
\newcommand{\abs}[1]{\left\vert #1 \right\vert} % for absolute value
\newcommand{\llangle}{\left\langle}
\newcommand{\rrangle}{\right\rangle}
\newcommand{\avg}[1]{\left< #1 \right>} % for average
\let\underdot=\d % rename builtin command \d{} to \underdot{}
\renewcommand{\d}{d}
\newcommand{\dder}[2]{\frac{d #1}{d #2}} % for derivatives
\newcommand{\ddder}[2]{\frac{d^2 #1}{d #2^2}} % for double derivatives
\newcommand{\pder}[2]{\frac{\partial #1}{\partial #2}}
% for partial derivatives
\newcommand{\pdd}[2]{\frac{\partial^2 #1}{\partial #2^2}}
% for double partial derivatives
\newcommand{\pdc}[3]{\left( \frac{\partial #1}{\partial #2}
 \right)_{#3}} % for thermodynamic partial derivatives
\newcommand{\ket}[1]{\left| #1 \right>} % for Dirac bras
\newcommand{\bra}[1]{\left< #1 \right|} % for Dirac kets
\newcommand{\braket}[2]{\left< #1 \vphantom{#2} \right|
 \left. #2 \vphantom{#1} \right>} % for Dirac brackets
\newcommand{\matrixel}[3]{\left< #1 \vphantom{#2#3} \right|
 #2 \left| #3 \vphantom{#1#2} \right>} % for Dirac matrix elements
\newcommand{\grad}[1]{\gv{\nabla} #1} % for gradient
\let\divsymb=\div % rename builtin command \div to \divsymb
\renewcommand{\div}[1]{\gv{\nabla} \cdot #1} % for divergence
\newcommand{\curl}[1]{\gv{\nabla} \times #1} % for curl
\let\baraccent=\= % rename builtin command \= to \baraccent
\renewcommand{\=}[1]{\stackrel{#1}{=}} % for putting numbers above =
\newcommand{\vhat}[1]{\ensuremath{\mathbf{\hat{#1}}}} % for vectors
\newcommand{\vvhat}[1]{\ensuremath{\bm{\hat{#1}}}} % for vectors

\usepackage{xcolor}
\definecolor{linkc}{RGB}{43,116,165}
\definecolor{ocre}{RGB}{243,102,25}
\definecolor{mybrown}{RGB}{128,64,0}


\definecolor{linkc}{RGB}{31,93,135}
\newcommand{\linkc}[1]{\textcolor{linkc}{#1}}
\newcommand{\linkcb}[1]{\textbf{\textcolor{linkc}{#1}}}


\usepackage{tcolorbox}

\mathchardef\mhyphen="2D

\tcbuselibrary{theorems}
\newtcolorbox{warning}{colback=mybrown!5!white,colframe=mybrown!45!white, title = Warning}

\newtcolorbox{attention}{colback=mybrown!5!white,colframe=mybrown!45!white}

\theoremstyle{plain}

\theoremstyle{definition}
\newtheorem*{definition}{Definition}%[section]
\newtheorem*{definitionT}{Note}%[section]
\usepackage[framemethod=default]{mdframed}
%\newmdenv[backgroundcolor=red]{tBox}
%\newmdenv[leftmargin=1cm,linecolor=blue]{aBox}
%\RequirePackage[framemethod=default]{mdframed}

\newtheorem*{theorem*}{Theorem}
\newtheorem{theorem}{Theorem}

\newmdenv[skipabove=12pt,
skipbelow=7pt,
rightline=false,
leftline=true,
topline=false,
bottomline=false,
linecolor=mybrown,
innerleftmargin=5pt,
innerrightmargin=5pt,
innertopmargin=10pt,
leftmargin=25pt,
rightmargin=0cm,
linewidth=4pt,
innerbottommargin=0pt]{dBox}

\newenvironment{note}{
\begin{dBox}
\begin{definitionT}}
{\end{definitionT}
\end{dBox}}

\begin{document}
\fontsize{5mm}{6mm}\selectfont\thispagestyle{empty}

\thispagestyle{empty}
\begin{center}
\textsc{Lecture 12}\\[1.5ex]
{\Huge FK5024: Particle and Nuclear Physics, Astrophysics and Cosmology\\}
\vspace{3mm}
{\large PART III: Astrophysics and Cosmology \\}
%\vspace{3mm}
Jon E. Gudmundsson\footnote{\href{http://jon.fysik.su.se}{\linkc{http://jon.fysik.su.se}}} \\
%\vspace{-3mm}
\linkc{jon@fysik.su.se}
\end{center}

Lecture 11 focused on some astrophysical connections to nuclear and particle physics. In the next few lectures, we will discuss basic concepts in astrophysics and cosmology. The coursebook for this part is Andrew R.\ Liddle's \textit{An Introduction to Cosmology} (2nd edition). If you are interested in learning more about this stuff, definitely take a look at courses such as FK7050, FK8025, AS5005, and AS7003.

\begin{attention}
This lecture should be supplemented by Liddle: 1-2, 5.1-5.2, 6.1, 3.1-3.3, 3.6
\end{attention}

\section{Distances Measurements}
It is fair to assume that a lot of people have looked up at the night sky and wondered about the stars. How far away are they? One way to estimate the distance to the stars is to make use of parallax angles. This approach relies on a basic observation about the sun-earth relationship: we are revolving around the sun in a roughly circular orbit. The distance to the sun has been known for quite some time. We can state that $r = 1.496 \times 10^{11}\:\mrm{m}$ (approximately 8 light minutes).

Figure \ref{fig:parallax} shows setup for distance measurement via parallax angle. The distance to a nearby star is $D$, the parallax angle is $p$, and the sun-earth distance is $r$. For small angles we can write
\begin{equation}
\sin (p) \approx p = \frac{r}{D} \: \Rightarrow \: D = \frac{r}{p}.
\end{equation}

%We can use this measurement to estimate the distance other nearby objects through a technique known as parallax measurement. Assuming we are on a circular orbit around the sun, we can use simple trigonometry to estimate the distance to far-away objects by measuring the change in their apparent angle over a 6-month period.

\begin{figure}[t]
\begin{center}
    \includegraphics*[angle=0,width=0.5\textwidth]{img/parallax.png}
    \caption[Insert text]{Representation of parallax angle to a distant object enabled by the non-negligible diameter of Earth's orbit (compared to the distance to the source). Parallax angles are only useful for determining distances to stars in our own galaxy.}
\label{fig:parallax}
\end{center}
\end{figure}

We find that some stars on the night-sky appear to move significantly over half a year, while other remain fixed. The stars that seem to move with respect to the fixed background are therefore closer to us than the rest.

Recently, the European Space Agency launched a satellite mission that mapped stars in our Galaxy using exactly this approach. The satellite mission, Gaia, has an angular resolution of approximately $25 \times 10^{-6}$ arcsec which effectively means that the experiment can measure distance to 20 million star with roughly 1\% accuracy.

For larger distances, the parallax angle becomes too small and we have to resort to different methods. One of this relies on so-called standard candles (see e.g.\ discussion about supernovas in future lectures).

\textbf{Definition:} \\
1 pc (parsec) is the distance which gives $p = 1 \ \mrm{arcsec}$ \\
$1 \ \mrm{deg} = \pi / 180 \ \mrm{rad}, 1 \ \mrm{arcmin} = 1/60 \ \mrm{deg}, 1 \ \mrm{arcsec} = 1/60 \ \mrm{arcmin}$,  \\
$\Rightarrow 1 \ \mrm{arcsec} = \pi / 180 / 3600 \ \mrm{rad} \approx 4.85 \times 10^{-6} \ \mrm{rad}$ \\ \\
\textbf{Question:} What is 1 pc in SI units (m)? \\
\textbf{Answer:}
\begin{equation}
D = \frac{d}{p} = \frac{1.496 \times 10^{11} }{4.85 \times 10^{-6}} = 3.09 \times 10^{16} \ \mrm{m}
\end{equation}

\section{Angular resolution}
It is important to understand how the properties of our telescopes impact angular resolution. Diffraction around a circular aperture is a concept that is discussed in most introductory physics textbooks. 

A plane wave illuminating a circular aperture of diameter $d$ will generate a diffraction pattern. This diffraction pattern is particularly strong if the dimensions of the circular aperture is similar in size to the wavelength of the incident electromagnetic radiation.

Figure \ref{fig:airy} shows the so-called Airy pattern which is described by 
\begin{equation}
I(\theta) = I(0) \left[ \frac{2J_1(kd\sin(\theta))}{kd\sin(\theta)} \right],
\end{equation}
where $J_1(\theta)$ is the Bessel function of the first kind and of order one, and $k = 2\pi/\lambda$ is the wavenumber. The relation between the first null of the diffraction pattern and the wavelength and dimensions of the aperture is known as the Rayleigh criterion
\begin{equation}
\sin (\theta) = 1.22 \lambda /d.
\end{equation}

\begin{figure}[t]
\begin{center}
    \includegraphics*[angle=0,width=0.8\textwidth]{img/airy_pattern.png}
    \caption[Insert text]{Airy pattern describing plane wave diffraction from a circular aperture with diameter $a$.}
\label{fig:airy}
\end{center}
\end{figure}


\section{Luminosity and magnitudes}
The the photosphere of the sun is at roughly $5780 \:\mrm{K}$. The radius of the sun is $R_\odot = 6.96 \times 10^{8} \:\mrm{m}$. According to the Stefan-Boltzmann law, the luminosity of the sun is found to be
\begin{equation}
L _\odot = 4\pi R_\odot ^{2} \times \sigma _\mrm{SB} T^{4} = 3.8 \times 10^{26} \:\mrm{W}.
\end{equation}
The flux, defined as the power per unit area, is therefore
\begin{equation}
F = \frac{L_\odot}{4\pi r^{2}} \approx 1350 \:\mrm{W/m}^{2}
\end{equation}
Compare this number with the expected output of a $1 \:\mrm{m}^{2}$ solar panel.

Astronomers tend to use magnitudes to describe the brightness of objects on the sky. The definition of apparent magnitude is 
\begin{equation}
m = -2.5 \log _{10} \left( \frac{F}{1 \mrm{W/m}^{2}} \right) + \mrm{const}.
\end{equation}
This approach to measuring brightness has some nice features. For example, we know that for Vega, a star that is 25 light-years away, we have $F_\odot / F_\mrm{vega} = 5\times 10^{10}$. From this we find that
\begin{equation}
m_\odot - m_V = -2.5 \log _{10} \left( \frac{F_\odot}{F_V} \right) = -2.5 \times 10.7 = -26.7.
\end{equation}
The relative magnitude of the sun compared to Vega is V = -26.7. 

\begin{note}
Relative magnitude is typically quoted for a particular frequency band or a filter which determines the sensitivity of your detector element as a function of frequency. Typical optical cameras that are used in astronomy have a relatively wide frequency sensitivity which is then truncated by the use of specifically designed filters. If you are into optical astronomy, you may have heard about U, B, V, R, I filters.
\end{note}

Relative scales in astronomy are useful because absolute measurements are complicated by things like the Earth's atmosphere (which can change depending on the time and frequency). We can observe objects on the sky over a wide range. For example, the Hubble space telescope is able to observe objects down to an apparent magnitude of about~$+30$.

\begin{figure}[t]
\begin{center}
    \includegraphics*[angle=0,width=0.5\textwidth]{img/cepheids.png}
    \caption[Insert text]{A typical cepheid light curve similar to the ones that Leavitt measured. The graph shows apparent magnitude as a function of time.}
\label{fig:cepheids}
\end{center}
\end{figure}

We can also define an absolute magnitude scale. This number represents the intrinsic luminosity of an object and it is therefore harder to measure. We can define the absolute magnitude as the magnitude exactly 10~pc away from the object (normally star).  Then
\begin{equation}
m_X - M = -2.5 \log _{10} \left( \frac{F_X}{F_M} \right) = -2.5 \log _{10} \left( \frac{(10 \:\mrm{pc})^{2}}{r^{2}_X} \right),
\end{equation}
which implies that
\begin{equation}
m_X - M = -2 \times 2.5 \log_{10} \left( \frac{10 \:\mrm{pc}}{r_X} \right) = 5 \log _{10} \left( \frac{r_x}{10 \: \mrm{pc}} \right).
\end{equation}
The absolute magnitude goes up by +5 when you increase the distance to the object by a factor of 10.

We can use these definitions to characterize the brightness of astrophysical objects with time. Henrietta Leavitt worked on cataloging optical measurements at the turn of the 20th century. She made a remarkable discovery when she noted a class of pulsating stars, now referred to as "cepheids", that varied by 1-2 in relative magnitude over a period of a few days in a very repeatable manner (see Figure \ref{fig:cepheids}).\footnote{Cepheids are named after the star $\delta$-Cephei in the constellation of the Cepheus.} Leavitt cataloged over 1500 variable stars in the Magellanic Clouds and discovered that brighter Cepheids take a longer time to vary (the oscillation period is longer).\footnote{The Magellanic clouds are dwarf satellite galaxies to our Milky way galaxy, roughly about 150,000 light years away from the Earth.} This fact is now partially used to calibrate the distance scale of our universe (see future lectures). %Figure \ref{fig:cepheids} shows a rough representation of the Cepheid magnitude as a function of time.

When combining data on multiple cepheids, we see a clear relation between their luminosity and the period of oscillation (see Figure \ref{fig:lr}).

\begin{figure}[t]
\begin{center}
    \includegraphics*[angle=0,width=0.6\textwidth]{img/luminosity_relation.png}
    \caption[Insert text]{Leavitt's law. A relation between the luminosity of cepheid variables and their period.}
\label{fig:lr}
\end{center}
\end{figure}

\section{The Hubble-Lemaitre law}
The Hubble-Lemaitre law describes an apparent increase in recession velocity with distance.\footnote{Most people still refer to this as simply the Hubble law, but the International Astronomical Union (IAU) recently voted to change the name to acknowledge Lemaitre's role in the formulation of a cosmological model that incorporate an expanding universe.} Simply put, the Hubble law is
\begin{equation}
v = H d
\end{equation}
where $v$ is the recessional velocity, $d$ is the proper distance to a particular galaxy (or distant object), and $H$ is some constant typically expressed in units of $\mrm{km /  s / Mpc}$. The measurements of Hubble and others showed that galaxies were predominantly receding away from us (the Milky Way) and that the velocity had a tendency to increase with distance. Figure \ref{fig:hubble} shows a famous graph published by Hubble in 1929 that roughly demonstrates this relationship. Hubble's early estimate put the constant at $H_0 = 500 \: \mrm{km /  s / Mpc}$. This has now been shown to be an overestimate, the more accurate value is closer to $H_0 = 70 \:\mrm{km /  s / Mpc}$. Before we talk further about the Hubble law, let's introduce the concept of redshift. 

\begin{figure}[t]
\begin{center}
    \includegraphics*[angle=0,width=0.6\textwidth]{img/hubble_relation.png}
    \caption[Insert text]{Graph published in a paper by Edwin Hubble in 1929 showing the velocity-distance relation for a few galaxies.}
\label{fig:hubble}
\end{center}
\end{figure}

The astrophysical redshift is a crucially important quantity in cosmology. Redshift is simply Doppler shifting of light, which on cosmological scales always leads to an increase in the wavelength of incoming radiation. The relativistic Doppler effect is:
\begin{equation}
\frac{\lambda _\mrm{obs}}{\lambda _\mrm{emit}} = 1 + z = \gamma \left(1 + \frac{v}{c} \right),
\end{equation}
where $\gamma$ is the so-called Lorentz boost factor
\begin{equation}
\gamma = \left( \sqrt{1 - \frac{v^2}{c^2}} \right)^{-1}
\end{equation}
The redshift is
\begin{equation}
z = \frac{\lambda _\mrm{obs} - \lambda _\mrm{lab}}{\lambda _\mrm{lab}} = \frac{\Delta \lambda}{\lambda _\mrm{lab}} \approx \frac{v}{c}.
\end{equation} 
For small velocities (compared to the speed of light), we can simply write:
\begin{equation}
v = c z,
\end{equation}
where $c$ is the speed of light in vacuum (roughly $2.998 \times 10^{8} \:\mrm{m/s}$). Spectroscopy allows to characterize and classify nearby stars. As we look at far-away objects, we notice that spectral fingerprints are shifted in wavelength space. 

%Taken at face value, Hubble's discovery implies that the Universe is expanding. 

\section{Some cosmological principles} 
\begin{figure}[t]
\begin{center}
    \includegraphics*[angle=0,width=0.8\textwidth]{img/hubble.png}
    \caption[Insert text]{The balloon analogy for Hubble expansion. To some extent, gravitationally bound systems are independent to the expansion of space. The comoving distance between clusters of galaxies remains constant while the physical distance changes with the scale factor.}
\label{fig:hubble_expansion}
\end{center}
\end{figure}
 
\textbf{Definitions:} The Copernican principle states that we, as observers of the universe, do not have the benefit of a privileged vantage point. The cosmological principle states that on large enough scales, the universe is both homogenous and isotropic. 
 
Homogenity: The universe looks the same at each point.

Isotropy: The universe looks the same in every direction.

Modern sky surveys suggest that our local universe is isotropic and homogenous over the largest scales that we can observer and we assume that this is a property of the universe as a whole.

 \begin{note}
 The popular literature contains a wealth of publications with brilliant descriptions of the early days of cosmology \cite{Lemaitre2005, Guth1997, Singh2005, Weinberg1993}. As astronomical observatories accumulated data, it became clear that the universe contained a large number of galaxies similar to our own. So far, nothing suggests our own galaxy is much different from the estimated hundreds of billions of galaxies in the observable universe. Similarly, our location in the Milky Way, our own galaxy, seems arbitrary. This gives some credence to the Copernican principle.
 \end{note}

In a universe that is expanding, it is helpful to introduce the concept that relates physical distances at any point in time to distances at a fixed reference time. We can write
\begin{equation}
r(t) = a(t) \chi
\end{equation}
where $r(t)$ is the physical distance at time $t$, $a(t)$ is a dimensionless scaling factor that represents the expansion of space with time, and $\chi$ is the so-called comoving distance which remains constant while space expands. The scale factor is a crucial concept in cosmology. 

Assuming an isotropic expansion of space we can write the velocity of recession as 
\begin{equation}
\bm{v} = d\bm{r} / dt.
\end{equation}
Since the expansion is along $\bm{r}$, we can write 
\begin{equation}
\bm{v} = \frac{|\dot{\bm{r}}|}{|\bm{r}|} \bm{r} = \frac{\dot{a}}{a} \bm{r},
\end{equation}
where we used the fact that $r = a\chi$ (the scale factor is driving the expansion). This is the Hubble law
\begin{equation}
\bm{v} = H\bm{r},
\end{equation}
where we see that 
\begin{equation}
H = \frac{\dot{a}}{a}.
\end{equation}
The value that we measure today is denoted with the subscript "0" and we write $H_0 = 70 \:\mrm{km/s/Mpc}$.

Let's assume that two galaxies were really close together at some point in the past ($d_\mrm{past} \ll d$). Then, assuming a universe that expands at a constant rate $H_0$, we can ask: At what time, $t$, did $d_\mrm{past} \to 0$? The answer to that question is simply 
\begin{equation}
t = \frac{d}{v_r} = \frac{d}{H_0 d} = \frac{1}{H_0}.
\end{equation}
Notice that this answer is independent of the present separation $d$. This suggests that at some time in the past, all galaxy pairs were close to each other. If we assume that $H_0 = 70 \mrm{km/s/Mpc}$ we will find that $1/H_0 = 13.97$~billion years. 



\section{Blackbody Radiation}
The Planck blackbody radiation formula plays a crucial role in cosmology
\begin{equation}
B(\nu, T) = \frac{2hf^3}{c^2} \frac{1}{e^\frac{hf}{k_\mrm{B}T} - 1} \quad [\mrm{W/m^2/Hz/sr}].
\label{eq:bb}
\end{equation}
The equation describes the spectral radiance of a blackbody as a function of frequency and the temperature of the blackbody. The universe is permiated with radiation that can be well characterized as that of a $2.7\:\mrm{K}$ blackbody. This is the so-called cosmic microwave background (CMB).

\begin{figure}[t]
\begin{center}
    \includegraphics*[angle=0,width=0.8\textwidth]{img/cmb_spectrum.png}
    \caption[Insert text]{Blackbody spectrum of the cosmic microwave background as measured by a number experiments \cite{Smoot1998}. Notice that a power law description of the spectral energy distribution fits the measurements quite well at low frequencies (this is the Rayleigh-Jeans limit).}
\label{fig:cmb_spectrum}
\end{center}
\end{figure}
 
Photons are bosons and as such there is no Pauli exclusion principle that prevents all photons from bunching in their lowest ground state. The occupation number per mode $N$ is given by the Planck function
\begin{equation}
N = \frac{1}{e^\frac{hf}{k_\mrm{B}T} - 1},
\end{equation}
where $k_\mrm{B} = 1.381 \times 10^{-23} \:\mrm{J/K} = 8.619 \times 10^{-5} \:\mrm{eV/K}$ is the Boltzmann constant. Note that photons can have two distinct polarizations states. 

In the limit when $hf \ll k_\mrm{B}T$ we have $\exp({hf/k_\mrm{B}T}) \approx 1 + hf/k_\mrm{B}T$ and Equation~\ref{eq:bb} reduces to
\begin{equation}
B(\nu, T) = \frac{2k_\mrm{B}T}{c^{2}}f^{2}.
\end{equation}
In this case, the spectral radiance of the blackbody is linearly dependent on temperature. This is the Rayleigh-Jeans limit of the Planck blackbody function.

\begingroup
\bibliographystyle{unsrtnat}
\linespread{0.5}\selectfont
\bibliography{lecture12}
\endgroup

\end{document}


