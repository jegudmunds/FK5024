% !TEX encoding = UTF-8 Unicode
% !TEX TS-program = pdflatexmk

\documentclass[a4paper,12pt]{article}

\usepackage[utf8]{inputenc}
\usepackage{geometry}
\usepackage{redefine-sections}
\usepackage{amsmath}
\usepackage{amsthm}
\usepackage{graphicx}
\usepackage{fancyhdr}
\usepackage{tikz}
\usepackage{pstricks}
\usepackage{pst-node}
\usepackage{wrapfig}
\usepackage{graphicx}
\usepackage{bibspacing}
\usepackage{multicol}
\usepackage{csquotes}
\usepackage[numbers,sort&compress]{natbib}
\usepackage{hyperref}
\usepackage{wrapfig}
\usepackage{bm}

\makeatletter

% Redefine maketitle
%
\def\maketitle{%
\par\textbf{\@title}%
\par{\@author}%
\par}

% Redefine \em and \emph
%
\DeclareRobustCommand{\em}{%
  \@nomath\em \if b\expandafter\@car\f@series\@nil
  \normalfont \else \bfseries \fi}

\makeatother

\geometry{left=2cm,right=2cm,top=2.5cm,bottom=2.5cm}
\lhead{\textsc{FK5024}}
\rhead{\textsc{Lecture 12}}
\pagestyle{fancy}

\theoremstyle{remark}
\newtheorem*{example}{Example}
\setlength{\parindent}{0pt}
\setlength{\parskip}{1.5em}
\renewcommand{\familydefault}{\sfdefault}

%  frequent science terms
\newcommand{\lcdm}{$\mathrm{\Lambda CDM}$ }
\newcommand{\lcdmn}{$\mathrm{\Lambda CDM}$}
\newcommand{\etal}{et al.\@ }
\newcommand{\etaln}{et al.\@}
\newcommand{\mrm}[1]{\mathrm{#1}}

%% From Header.tex, see tmp/
%% http://www.dfcd.net/articles/latex/latex.html
\renewcommand{\v}[1]{\ensuremath{\mathbf{#1}}} % for vectors
\newcommand{\vv}[1]{\ensuremath{\vec{\mathbf{#1}}}} % for vectors2
\newcommand{\vvv}[1]{\ensuremath{{\bm{#1}}}} % for vectors2
\newcommand{\gv}[1]{\ensuremath{\mbox{\boldmath$ #1 $}}}
\newcommand{\ellp}{\ell '}
\newcommand{\uv}[1]{\ensuremath{\mathbf{\hat{#1}}}} % for unit vector
\newcommand{\abs}[1]{\left\vert #1 \right\vert} % for absolute value
\newcommand{\llangle}{\left\langle}
\newcommand{\rrangle}{\right\rangle}
\newcommand{\avg}[1]{\left< #1 \right>} % for average
\let\underdot=\d % rename builtin command \d{} to \underdot{}
\renewcommand{\d}{d}
\newcommand{\dder}[2]{\frac{d #1}{d #2}} % for derivatives
\newcommand{\ddder}[2]{\frac{d^2 #1}{d #2^2}} % for double derivatives
\newcommand{\pder}[2]{\frac{\partial #1}{\partial #2}}
% for partial derivatives
\newcommand{\pdd}[2]{\frac{\partial^2 #1}{\partial #2^2}}
% for double partial derivatives
\newcommand{\pdc}[3]{\left( \frac{\partial #1}{\partial #2}
 \right)_{#3}} % for thermodynamic partial derivatives
\newcommand{\ket}[1]{\left| #1 \right>} % for Dirac bras
\newcommand{\bra}[1]{\left< #1 \right|} % for Dirac kets
\newcommand{\braket}[2]{\left< #1 \vphantom{#2} \right|
 \left. #2 \vphantom{#1} \right>} % for Dirac brackets
\newcommand{\matrixel}[3]{\left< #1 \vphantom{#2#3} \right|
 #2 \left| #3 \vphantom{#1#2} \right>} % for Dirac matrix elements
\newcommand{\grad}[1]{\gv{\nabla} #1} % for gradient
\let\divsymb=\div % rename builtin command \div to \divsymb
\renewcommand{\div}[1]{\gv{\nabla} \cdot #1} % for divergence
\newcommand{\curl}[1]{\gv{\nabla} \times #1} % for curl
\let\baraccent=\= % rename builtin command \= to \baraccent
\renewcommand{\=}[1]{\stackrel{#1}{=}} % for putting numbers above =
\newcommand{\vhat}[1]{\ensuremath{\mathbf{\hat{#1}}}} % for vectors
\newcommand{\vvhat}[1]{\ensuremath{\bm{\hat{#1}}}} % for vectors

\usepackage{xcolor}
\definecolor{linkc}{RGB}{43,116,165}
\definecolor{ocre}{RGB}{243,102,25}
\definecolor{mybrown}{RGB}{128,64,0}


\definecolor{linkc}{RGB}{31,93,135}
\newcommand{\linkc}[1]{\textcolor{linkc}{#1}}
\newcommand{\linkcb}[1]{\textbf{\textcolor{linkc}{#1}}}


\usepackage{tcolorbox}

\mathchardef\mhyphen="2D

\tcbuselibrary{theorems}
\newtcolorbox{warning}{colback=mybrown!5!white,colframe=mybrown!45!white, title = Warning}

\newtcolorbox{attention}{colback=mybrown!5!white,colframe=mybrown!45!white}

\theoremstyle{plain}

\theoremstyle{definition}
\newtheorem*{definition}{Definition}%[section]
\newtheorem*{definitionT}{Note}%[section]
\usepackage[framemethod=default]{mdframed}
%\newmdenv[backgroundcolor=red]{tBox}
%\newmdenv[leftmargin=1cm,linecolor=blue]{aBox}
%\RequirePackage[framemethod=default]{mdframed}

\newtheorem*{theorem*}{Theorem}
\newtheorem{theorem}{Theorem}

\newmdenv[skipabove=12pt,
skipbelow=7pt,
rightline=false,
leftline=true,
topline=false,
bottomline=false,
linecolor=mybrown,
innerleftmargin=5pt,
innerrightmargin=5pt,
innertopmargin=10pt,
leftmargin=25pt,
rightmargin=0cm,
linewidth=4pt,
innerbottommargin=0pt]{dBox}

\newenvironment{note}{
\begin{dBox}
\begin{definitionT}}
{\end{definitionT}
\end{dBox}}

\begin{document}
\fontsize{5mm}{6mm}\selectfont\thispagestyle{empty}

\thispagestyle{empty}
\begin{center}
\textsc{Lecture 12}\\[1.5ex]
{\Huge FK5024: Particle and Nuclear Physics, Astrophysics and Cosmology\\}
\vspace{3mm}
{\large PART III: Astrophysics and Cosmology \\}
%\vspace{3mm}
Jon E. Gudmundsson\footnote{\href{http://jon.fysik.su.se}{\linkc{http://jon.fysik.su.se}}} \\
%\vspace{-3mm}
\linkc{jon@fysik.su.se}
\end{center}

Lecture 11 focused on some astrophysical connections to nuclear and particle physics. In the next few lectures, we will discuss basic concepts in astrophysics and cosmology. The coursebook for this part is Andrew R.\ Liddle's \textit{An Introduction to Cosmology} (2nd edition). If you are interested in learning more about this stuff, definitely take a look at courses such as FK7050, FK8025, AS5005, and AS7003.

\begin{attention}
This lecture should be supplemented by Liddle: 1-2, 5.1-5.2, 6.1, 3.1-3.3, 3.6
\end{attention}

\section{Distances Measurements}
It is fair to assume that a lot of people have looked up at the night sky and wondered about the stars. How far away are they? One way to estimate the distance to the stars is to make use of parallax angles. This approach relies on a basic observation about the sun-earth relationship: we are revolving around the sun in a roughly circular orbit. The distance to the sun has been known for quite some time. We can state that $r = 1.496 \times 10^{11}\:\mrm{m}$ (approximately 8 light minutes).

Figure \ref{fig:parallax} shows setup for distance measurement via parallax angle. The distance to a nearby star is $D$, the parallax angle is $p$, and the sun-earth distance is $r$. For small angles we can write
\begin{equation}
\sin (p) \approx p = \frac{r}{D} \: \Rightarrow \: D = \frac{r}{p}.
\end{equation}

%We can use this measurement to estimate the distance other nearby objects through a technique known as parallax measurement. Assuming we are on a circular orbit around the sun, we can use simple trigonometry to estimate the distance to far-away objects by measuring the change in their apparent angle over a 6-month period.

\begin{figure}[t]
\begin{center}
    \includegraphics*[angle=0,width=0.5\textwidth]{img/parallax.png}
    \caption[Insert text]{Representation of parallax angle to a distant object enabled by the non-negligible diameter of Earth's orbit (compared to the distance to the source). Parallax angles are only useful for determining distances to stars in our own galaxy.}
\label{fig:parallax}
\end{center}
\end{figure}

We find that some stars on the night-sky appear to move significantly over half a year, while other remain fixed. The stars that seem to move with respect to the fixed background are therefore closer to us than the rest.

Recently, the European Space Agency launched a satellite mission that mapped stars in our Galaxy using exactly this approach. The satellite mission, Gaia, has an angular resolution of approximately $25 \times 10^{-6}$ arcsec which effectively means that the experiment can measure distance to 20 million star with roughly 1\% accuracy.

For larger distances, the parallax angle becomes too small and we have to resort to different methods. One of this relies on so-called standard candles (see e.g.\ discussion about supernovas in future lectures).

\textbf{Definition:} \\
1 pc (parsec) is the distance which gives $p = 1 \ \mrm{arcsec}$ \\
$1 \ \mrm{deg} = \pi / 180 \ \mrm{rad}, 1 \ \mrm{arcmin} = 1/60 \ \mrm{deg}, 1 \ \mrm{arcsec} = 1/60 \ \mrm{arcmin}$,  \\
$\Rightarrow 1 \ \mrm{arcsec} = \pi / 180 / 3600 \ \mrm{rad} \approx 4.85 \times 10^{-6} \ \mrm{rad}$ \\ \\
\textbf{Question:} What is 1 pc in SI units (m)? \\
\textbf{Answer:}
\begin{equation}
D = \frac{d}{p} = \frac{1.496 \times 10^{11} }{4.85 \times 10^{-6}} = 3.09 \times 10^{16} \ \mrm{m}
\end{equation}

\section{Angular resolution}
It is important to understand how the properties of our telescopes impact angular resolution. Diffraction around a circular aperture is a concept that is discussed in most introductory physics textbooks. 

A plane wave illuminating a circular aperture of diameter $d$ will generate a diffraction pattern. This diffraction pattern is particularly strong if the dimensions of the circular aperture is similar in size to the wavelength of the incident electromagnetic radiation.

Figure \ref{fig:airy} shows the so-called Airy pattern which is described by 
\begin{equation}
I(\theta) = I(0) \left[ \frac{2J_1(kd\sin(\theta))}{kd\sin(\theta)} \right],
\end{equation}
where $J_1(\theta)$ is the Bessel function of the first kind and of order one, and $k = 2\pi/\lambda$ is the wavenumber. The relation between the first null of the diffraction pattern and the wavelength and dimensions of the aperture is known as the Rayleigh criterion
\begin{equation}
\sin (\theta) = 1.22 \lambda /d.
\end{equation}

\begin{figure}[t]
\begin{center}
    \includegraphics*[angle=0,width=0.8\textwidth]{img/airy_pattern.png}
    \caption[Insert text]{Airy pattern describing plane wave diffraction from a circular aperture with diameter $a$.}
\label{fig:airy}
\end{center}
\end{figure}


\section{Luminosity and magnitudes}
The the photosphere of the sun is at roughly $5780 \:\mrm{K}$. The radius of the sun is $R_\odot = 6.96 \times 10^{8} \:\mrm{m}$. According to the Stefan-Boltzmann law, the luminosity of the sun is found to be
\begin{equation}
L _\odot = 4\pi R_\odot ^{2} \times \sigma _\mrm{SB} T^{4} = 3.8 \times 10^{26} \:\mrm{W}.
\end{equation}
The flux, defined as the power per unit area, is therefore
\begin{equation}
F = \frac{L_\odot}{4\pi r^{2}} \approx 1350 \:\mrm{W/m}^{2}
\end{equation}
Compare this number with the expected output of a $1 \:\mrm{m}^{2}$ solar panel.

Astronomers tend to use magnitudes to describe the brightness of objects on the sky. The definition of apparent magnitude is 
\begin{equation}
m = -2.5 \log _{10} \left( \frac{F}{1 \mrm{W/m}^{2}} \right) + \mrm{const}.
\end{equation}
This approach to measuring brightness has some nice features. For example, we know that for Vega, a star that is 25 light-years away, we have $F_\odot / F_\mrm{vega} = 5\times 10^{10}$. From this we find that
\begin{equation}
m_\odot - m_V = -2.5 \log _{10} \left( \frac{F_\odot}{F_V} \right) = -2.5 \times 10.7 = -26.7.
\end{equation}
The relative magnitude of the sun compared to Vega is V = -26.7. 

\begin{note}
Relative magnitude is typically quoted for a particular frequency band or a filter which determines the sensitivity of your detector element as a function of frequency. Typical optical cameras that are used in astronomy have a relatively wide frequency sensitivity which is then truncated by the use of specifically designed filters. If you are into optical astronomy, you may have heard about U, B, V, R, I filters.
\end{note}

Relative scales in astronomy are useful because absolute measurements are complicated by things like the Earth's atmosphere (which can change depending on the time and frequency). We can observe objects on the sky over a wide range. For example, the Hubble space telescope is able to observe objects down to an apparent magnitude of about~$+30$.

\begin{figure}[t]
\begin{center}
    \includegraphics*[angle=0,width=0.5\textwidth]{img/cepheids.png}
    \caption[Insert text]{A typical cepheid light curve similar to the ones that Leavitt measured. The graph shows apparent magnitude as a function of time.}
\label{fig:cepheids}
\end{center}
\end{figure}

We can also define an absolute magnitude scale. This number represents the intrinsic luminosity of an object and it is therefore harder to measure. We can define the absolute magnitude as the magnitude exactly 10~pc away from the object (normally star).  Then
\begin{equation}
m_X - M = -2.5 \log _{10} \left( \frac{F_X}{F_M} \right) = -2.5 \log _{10} \left( \frac{(10 \:\mrm{pc})^{2}}{r^{2}_X} \right),
\end{equation}
which implies that
\begin{equation}
m_X - M = -2 \times 2.5 \log_{10} \left( \frac{10 \:\mrm{pc}}{r_X} \right) = 5 \log _{10} \left( \frac{r_x}{10 \: \mrm{pc}} \right).
\end{equation}
The absolute magnitude goes up by +5 when you increase the distance to the object by a factor of 10.

We can use these definitions to characterize the brightness of astrophysical objects with time. Henrietta Leavitt worked on cataloging optical measurements at the turn of the 20th century. She made a remarkable discovery when she noted a class of pulsating stars, now referred to as "cepheids", that varied by 1-2 in relative magnitude over a period of a few days in a very repeatable manner (see Figure \ref{fig:cepheids}).\footnote{Cepheids are named after the star $\delta$-Cephei in the constellation of the Cepheus.} Leavitt cataloged over 1500 variable stars in the Magellanic Clouds and discovered that brighter Cepheids take a longer time to vary (the oscillation period is longer).\footnote{The Magellanic clouds are dwarf satellite galaxies to our Milky way galaxy, roughly about 150,000 light years away from the Earth.} This fact is now partially used to calibrate the distance scale of our universe (see future lectures). %Figure \ref{fig:cepheids} shows a rough representation of the Cepheid magnitude as a function of time.

When combining data on multiple cepheids, we see a clear relation between their luminosity and the period of oscillation (see Figure \ref{fig:lr}).

\begin{figure}[t]
\begin{center}
    \includegraphics*[angle=0,width=0.6\textwidth]{img/luminosity_relation.png}
    \caption[Insert text]{Leavitt's law. A relation between the luminosity of cepheid variables and their period.}
\label{fig:lr}
\end{center}
\end{figure}

\section{The Hubble-Lemaitre law}
The Hubble-Lemaitre law describes an apparent increase in recession velocity with distance.\footnote{Most people still refer to this as simply the Hubble law, but the International Astronomical Union (IAU) recently voted to change the name to acknowledge Lemaitre's role in the formulation of a cosmological model that incorporate an expanding universe.} Simply put, the Hubble law is
\begin{equation}
v = H d
\end{equation}
where $v$ is the recessional velocity, $d$ is the proper distance to a particular galaxy (or distant object), and $H$ is some constant typically expressed in units of $\mrm{km /  s / Mpc}$. The measurements of Hubble and others showed that galaxies were predominantly receding away from us (the Milky Way) and that the velocity had a tendency to increase with distance. Figure \ref{fig:hubble} shows a famous graph published by Hubble in 1929 that roughly demonstrates this relationship. Hubble's early estimate put the constant at $H_0 = 500 \: \mrm{km /  s / Mpc}$. This has now been shown to be an overestimate, the more accurate value is closer to $H_0 = 70 \:\mrm{km /  s / Mpc}$. Before we talk further about the Hubble law, let's introduce the concept of redshift. 

\begin{figure}[t]
\begin{center}
    \includegraphics*[angle=0,width=0.6\textwidth]{img/hubble_relation.png}
    \caption[Insert text]{Graph published in a paper by Edwin Hubble in 1929 showing the velocity-distance relation for a few galaxies.}
\label{fig:hubble}
\end{center}
\end{figure}

The astrophysical redshift is a crucially important quantity in cosmology. Redshift is simply Doppler shifting of light, which on cosmological scales always leads to an increase in the wavelength of incoming radiation. The relativistic Doppler effect is:
\begin{equation}
\frac{\lambda _\mrm{obs}}{\lambda _\mrm{emit}} = 1 + z = \gamma \left(1 + \frac{v}{c} \right),
\end{equation}
where $\gamma$ is the so-called Lorentz boost factor
\begin{equation}
\gamma = \left( \sqrt{1 - \frac{v^2}{c^2}} \right)^{-1}
\end{equation}
The redshift is
\begin{equation}
z = \frac{\lambda _\mrm{obs} - \lambda _\mrm{lab}}{\lambda _\mrm{lab}} = \frac{\Delta \lambda}{\lambda _\mrm{lab}} \approx \frac{v}{c}.
\end{equation} 
For small velocities (compared to the speed of light), we can simply write:
\begin{equation}
v = c z,
\end{equation}
where $c$ is the speed of light in vacuum (roughly $2.998 \times 10^{8} \:\mrm{m/s}$). Spectroscopy allows to characterize and classify nearby stars. As we look at far-away objects, we notice that spectral fingerprints are shifted in wavelength space. 

%Taken at face value, Hubble's discovery implies that the Universe is expanding. 

\section{Some cosmological principles} 
\begin{figure}[t]
\begin{center}
    \includegraphics*[angle=0,width=0.8\textwidth]{img/hubble.png}
    \caption[Insert text]{The balloon analogy for Hubble expansion. To some extent, gravitationally bound systems are independent to the expansion of space. The comoving distance between clusters of galaxies remains constant while the physical distance changes with the scale factor.}
\label{fig:hubble_expansion}
\end{center}
\end{figure}
 
\textbf{Definitions:} The Copernican principle states that we, as observers of the universe, do not have the benefit of a privileged vantage point. The cosmological principle states that on large enough scales, the universe is both homogenous and isotropic. 
 
Homogenity: The universe looks the same at each point.

Isotropy: The universe looks the same in every direction.

Modern sky surveys suggest that our local universe is isotropic and homogenous over the largest scales that we can observer and we assume that this is a property of the universe as a whole.

 \begin{note}
 The popular literature contains a wealth of publications with brilliant descriptions of the early days of cosmology \cite{Lemaitre2005, Guth1997, Singh2005, Weinberg1993}. As astronomical observatories accumulated data, it became clear that the universe contained a large number of galaxies similar to our own. So far, nothing suggests our own galaxy is much different from the estimated hundreds of billions of galaxies in the observable universe. Similarly, our location in the Milky Way, our own galaxy, seems arbitrary. This gives some credence to the Copernican principle.
 \end{note}

In a universe that is expanding, it is helpful to introduce the concept that relates physical distances at any point in time to distances at a fixed reference time. We can write
\begin{equation}
r(t) = a(t) \chi
\end{equation}
where $r(t)$ is the physical distance at time $t$, $a(t)$ is a dimensionless scaling factor that represents the expansion of space with time, and $\chi$ is the so-called comoving distance which remains constant while space expands. The scale factor is a crucial concept in cosmology. 

Assuming an isotropic expansion of space we can write the velocity of recession as 
\begin{equation}
\bm{v} = d\bm{r} / dt.
\end{equation}
Since the expansion is along $\bm{r}$, we can write 
\begin{equation}
\bm{v} = \frac{|\dot{\bm{r}}|}{|\bm{r}|} \bm{r} = \frac{\dot{a}}{a} \bm{r},
\end{equation}
where we used the fact that $r = a\chi$ (the scale factor is driving the expansion). This is the Hubble law
\begin{equation}
\bm{v} = H\bm{r},
\end{equation}
where we see that 
\begin{equation}
H = \frac{\dot{a}}{a}.
\end{equation}
The value that we measure today is denoted with the subscript "0" and we write $H_0 = 70 \:\mrm{km/s/Mpc}$.

Let's assume that two galaxies were really close together at some point in the past ($d_\mrm{past} \ll d$). Then, assuming a universe that expands at a constant rate $H_0$, we can ask: At what time, $t$, did $d_\mrm{past} \to 0$? The answer to that question is simply 
\begin{equation}
t = \frac{d}{v_r} = \frac{d}{H_0 d} = \frac{1}{H_0}.
\end{equation}
Notice that this answer is independent of the present separation $d$. This suggests that at some time in the past, all galaxy pairs were close to each other. If we assume that $H_0 = 70 \mrm{km/s/Mpc}$ we will find that $1/H_0 = 13.97$~billion years. 



\section{Blackbody Radiation}
The Planck blackbody radiation formula plays a crucial role in cosmology
\begin{equation}
B(\nu, T) = \frac{2hf^3}{c^2} \frac{1}{e^\frac{hf}{k_\mrm{B}T} - 1} \quad [\mrm{W/m^2/Hz/sr}].
\label{eq:bb}
\end{equation}
The equation describes the spectral radiance of a blackbody as a function of frequency and the temperature of the blackbody. The universe is permiated with radiation that can be well characterized as that of a $2.7\:\mrm{K}$ blackbody. This is the so-called cosmic microwave background (CMB).

\begin{figure}[t]
\begin{center}
    \includegraphics*[angle=0,width=0.8\textwidth]{img/cmb_spectrum.png}
    \caption[Insert text]{Blackbody spectrum of the cosmic microwave background as measured by a number experiments \cite{Smoot1998}. Notice that a power law description of the spectral energy distribution fits the measurements quite well at low frequencies (this is the Rayleigh-Jeans limit).}
\label{fig:cmb_spectrum}
\end{center}
\end{figure}
 

Photons are bosons and as such there is no Pauli exclusion principle that prevents all photons from bunching in their lowest ground state. The occupation number per mode $N$ is given by the Planck function
\begin{equation}
N = \frac{1}{e^\frac{hf}{k_\mrm{B}T} - 1},
\end{equation}
where $k_\mrm{B} = 1.381 \times 10^{-23} \:\mrm{J/K} = 8.619 \times 10^{-5} \:\mrm{eV/K}$ is the Boltzmann constant. Note that photons can have two distinct polarizations states. 

In the limit when $hf \ll k_\mrm{B}T$ we have $\exp({hf/k_\mrm{B}T}) \approx 1 + hf/k_\mrm{B}T$ and Equation~\ref{eq:bb} reduces to
\begin{equation}
B(\nu, T) = \frac{2k_\mrm{B}T}{c^{2}}f^{2}.
\end{equation}
In this case, the spectral radiance of the blackbody is linearly dependent on temperature. This is the Rayleigh-Jeans limit of the Planck blackbody function.

%\section{Important concepts}
%Here are some important concepts and/or phenomena to take away from the course material that are not covered in these lecture notes:
%\vspace{-7mm}
%\begin{itemize}
%\item Stars
%\item Galaxies
%\item Local group
%\item Galaxy clusters
%\item Large-scale smoothness of the universe
%\item Microwave, infrared, x-ray, and radio wavelength radiation
%\end{itemize}



 
%The cosmic microwave background (CMB) is an isotropic blackbody radiation field with temperature
%\begin{equation}
%T_\mrm{CMB} = (2.726 \pm 0.001) \,\mrm{K}.
%\end{equation}
%%Figure \ref{} shows measurements from the \cobe satellite and a best fit theoretical curve.
%
%%Remember that Max Planck wrote down the equation describing blackbody radiation
%
%%\begin{equation}
%%B(\nu,T) = \frac{2h\nu^3}{c^2}\frac{1}{e^{h\nu/kT}-1},
%%\end{equation}
%
%\section*{A Bit of History}
%%% Princeton / Penzias + Wilson
%%\textit{Stuff from this section won't be on the exam!}
%
%In the 1960's a group of physicists at Princeton University began to search for the presence of thermal radiation remaining from the ``primordial fireball.'' Their research was directed by Robert Dicke, who by then had made contributions to a wide range of physics, including radar development and atomic theory, but focused now on gravitation theory. A member of that group, Jim Peebles, independently derived the results of Alpher and Herman \cite{Dicke1965b}. Using microwave receiver technology that Dicke had developed twenty years earlier, called the Dicke radiometer, the Princeton group set about measuring this relic radiation on the roof of Guyot Hall in 1964. The experimental effort was led by Roll and Wilkinson \cite{FBB}. As the Princeton group was commencing its measurements another New Jersey duo, Penzias and Wilson, had begun using a Dicke radiometer for radio astronomy. Battling an unknown noise contaminant, the pair had exhausted all avenues of reason, as they resorted to the sweeping of pigeon droppings inside their monstrous receiver horn. The Bell Labs researchers eventually made contact with the Princeton group which helped them understand their predicament. Penzias and Wilson had serendipitously discovered the cosmic microwave background, an incredibly uniform blackbody signal coming from all directions on the sky \cite{Penzias1965}. Subsequent work by physicists at Princeton helped define the results and their theoretical implications \cite{Roll1966, Dicke1965}. The discovery of the CMB brought the big bang universe to the forefront of modern physics.
%
%\begin{figure}[t]
%\begin{center}
%    \includegraphics*[angle=0,width=0.8\textwidth]{img/peebles_plot.png}
%    \caption[The spectrum of the CMB]{The results of the two New Jersey measurements, published in 1966 \cite{Roll1966}, showing intensity as a function of wavelength with a 3~K blackbody spectrum plotted for comparison. Both the Bell Labs (7.35~cm) and the Princeton measurement (3.2~cm) were performed safely within the Rayleigh-Jeans limit, almost two orders of magnitude below the peak value of a 3~K blackbody. Figure reproduced courtesy of P.~J.~E.~Peebles.}
%\label{fig:peebles_plot}
%\end{center}
%\end{figure}
%
%Figure~\ref{fig:peebles_plot} shows the results from the first Bell Labs and Princeton measurements overlaid on a 3~K blackbody spectrum. Despite common belief, the 1964 Bell Labs measurements did not represent the first evidence for a uniform cosmic afterglow. The study of CN molecular spectra, published as early as 1940, suggested ``a maximum effective temperature of interstellar space'' of about~1--3~K~\cite{McKellar1940,Herzberg1950} and an excess temperature of space was reported during the commissioning of the Bell Labs receiver \cite{Ohm1961,JonesThesis}. Regardless of who should be acknowledged for the initial discovery, the study of this signal continued, and we now know that the CMB is almost a perfect blackbody with temperature $T_{\mrm{CMB}} = 2.726\:\mrm{K}$ \cite{Fixsen2009}. Its spectral radiance as a function of frequency, $\nu$, follows the form
%\begin{equation}
%B(\nu) = \frac{2h\nu^{3}}{c^{2}}\frac{1}{\exp(h\nu /k_{\mrm{B}}T) -1},
%\end{equation}
%where $h$ and $k_{\mrm{B}}$ are Planck and Boltzmann constants respectively, $T$ is the blackbody temperature, and $c$ is the speed of light in vacuum.

%%% Dark matter
%During these early days of cosmology, indirect evidence for the existence of a dark matter component had begun to emerge. Dark matter is hypothesized matter which interacts gravitationally with normal matter, but does not absorb or emit electromagnetic radiation. The first reference to dark matter was made by astronomer Fritz Zwicky, who observed that galaxies in the Coma Cluster have peculiar velocities which are inconsistent with velocities predicted by the virial theorem, given the estimated mass of visible matter \cite{Zwicky1933,Zwicky1937}. Observations of spiral galaxy rotation curves in the 1970's, notably by Vera Rubin \cite{Rubin1978,Rubin1980}, showed that angular velocities outside the galactic bulges were much larger than expected, given mass estimates. The predicament of those times is neatly summarized by the first sentence of a 1974 paper by Ostriker \etal \cite{Ostriker1974}:
%\begin{quote}
%\enquote{\textit{There are reasons, increasing in number and quality, to believe that the masses of ordinary galaxies may have been underestimated by a factor of 10 or more.}}
%\end{quote}
%More recent measurements suggest the ratio of dark matter to ordinary matter is about six to one. Unlike ordinary matter, dark matter is believed to reside in halos that are often concentric with galactic centers.\footnote{Dark matter halos are normally assumed to be spherical, not disk shaped like the word might suggest.} The presence of dark matter is seen not only in the velocity profiles of galaxies but also by mapping the peculiar motion of clustering galaxies. By mapping the phase space of galaxies we learn about the growth of structure in the universe and the governing dynamics. Perhaps the most beguiling evidence for dark matter is found in the beautiful composite images of the bullet cluster \cite{Markevitch2004}.
%
%The popular literature contains a wealth of publications with brilliant descriptions of the early days of cosmology \cite{Lemaitre2005,Guth1997,Singh2005,Weinberg1993}. As astronomical observatories accumulated data, it became clear that the universe contained a large number of galaxies similar to our own. So far, nothing suggests our own galaxy is much different from the estimated hundreds of billions of galaxies in the observable universe. Similarly, our location in the Milky Way, our own galaxy, seems arbitrary. This gives some credence to the Copernican principle.
%
%%% Nomenclature, paradoxes, etc.
%The Copernican principle states that we, as observers of the universe, do not have the benefit of a privileged vantage point. As modern sky surveys suggest that our local universe is isotropic and homogenous over the largest scales we conclude that so is the universe as a whole. Upon closer inspection we notice, however, that galaxies tend to clump together in halos with great lifeless voids in between. Figure~\ref{fig:2dfs} shows the distribution of approximately a hundred thousand galaxies across two patches on the sky as measured by the 2dF Galaxy Redshift Survey \cite{Colless2001}. Similar distributions can only be seen in simulations that include a dark matter component that dominates baryonic energy densities at the ratio of six to one \cite{Springel2005,Rasera2013}.
%\begin{figure}[t]
%\begin{center}
%    \includegraphics*[angle=0,width=0.9\textwidth]{img/slices_3deg_big_bw.png}
%    \caption[Results from the 2dF Galaxy Redshift Survey]{Results from the 2dF Galaxy Redshift Survey showing the distribution of galaxies projected onto the plane. Reproduced with permission from the 2dF Galaxy Redshift Survey Team \cite{Colless2001,Colless2014}.}\label{fig:2dfs}
%\end{center}
%\end{figure}
%

%
%%% COBE results
%Penzias' and Wilson's discovery of the cosmic microwave background, which later won them a Nobel prize, seeded the main bough of observational cosmology. Numerous experimental endeavors ensued, with efforts attempting to measure the uniformity \cite{Lubin1985} or spectral shape \cite{Hayakawa1987} of this fossil signal. Arguably the most famous of these early experiments is the Cosmic Background Explorer (\coben) satellite. The satellite was launched in 1989 with three instruments designed to measure different properties of the cosmic microwave background.
%
%Using only 9 minutes of data spanning wavelengths of 1~cm to 0.5~mm, the FIRAS instrument measured a background radiation which was fit well by a 2.7~K blackbody spectrum \cite{Mather1990}. During a 1992 meeting of the American Physical Society, measurements of CMB anisotropies were revealed, causing much stir in the scientific community. Publications of the main results followed in the Astrophysical Journal \cite{Fixsen1994,Mather1994}. After four years of observation, the coarse resolution DMR instrument had constructed a full sky image of the cosmic microwave background with fluctuation in the temperature of about ten parts per million.\footnote{Only the CMB dipole anisotropy had been measured before the DMR results.} This was, and continues to be, the strongest argument for isotropy and the Copernican principle (see Section \ref{sec:homo}). At this point, observational cosmology had been established as an avenue for answering fundamental questions about the nature of the universe. The hot big bang scenario was no longer contested.
%
%\section{Primordial Perturbations}
%
%%% Perturbations
%The average density of our universe corresponds to a proton per cubic meter! Yet, this seemingly lifeless universe is able to facilitate the growth of galaxies, solar systems, and planets --- structure as it is known to us. To explain this we invoke primordial density perturbations as quantum fluctuations of the inflaton (more on this during Lecture 11), or more generally as perturbations to the FLRW metric. We believe that perturbations, seeded during the very early universe, survive the transition from our speculative models of the embryonic universe to a largely uncontested hot and dense infant universe developing according to established laws of physics. We think that the perturbations are imprinted into the spectrum of the cosmic microwave background, and with proper care, can shed light on the early universe. By poking at the statistical properties of the CMB, we hope to learn something about the initial conditions of the universe.
%
%On large scales, variation in the temperature of the CMB are affected by spatial curvature perturbations, $\Phi$. On large scales, the CMB temperature anisotropies at some location on the sky, $\v{n}$, are connected to the gravitational effects of density perturbations
%\begin{equation}
%\frac{\delta T}{T}(\v{n}) = -\frac{1}{3}\Phi(\v{n}),
%\label{eq:sw}
%\end{equation}
%where the stuff on the right hand side represents potential energy (see Sections 11.2--11.6 in B\&G).
%
%The time-development of variations in matter, photons, energy, etc., is described by the Friedmann and fluid equations. The important thing to note is that variations in photon and matter densities are coupled: Gravitational attractions in the photon-baryon plasma tend to form halos with infalling matter while increasing photon pressure impedes this process and erases anisotropies. Different modes of compression and rarefication develop at the speed of sound in the plasma \cite{Hu1996}. This tug of war, referred to as acoustic oscillations, continues until the universe becomes electrically neutral. %has expanded enough, and therefore decreased in temperature, for protons and electrons to form hydrogen. % --- photons travel on geodesics --- and that gravitational infall is resisted by photon pressure radiating outwards.
%%% Temperature Anisotropies
%
%
%\section{Recombination}
%
%This event is called recombination and happens at a redshift of $z = 1090$, approximately 380,000 years after the big bang \cite{Zeldovich1969}. The event defines a veil at the edge of our horizon, sometimes referred to as the last scattering surface. The universe now becomes comparatively transparent to light and the photons stream freely in every direction with a fraction bombarding our telescopes today. These are the CMB photons, and they carry with them information about the fundamental oscillation modes at the time of recombination.
%
%Prior to recombination, photons and baryons are tightly coupled. Photons scatter off electrons and hydrogen atoms are quickly split apart
%\begin{align}
%\gamma + e^- &\longrightarrow \gamma + e^- \\
%H + \gamma &\longleftrightarrow p + e^-
%\end{align}
%%via Thomson scattering, which has a cross section
%%\begin{equation}
%%\sigma _e = \frac{8\pi}{3}\left( \frac{\alpha \hbar}{m_e c} \right)^2 = 6.65 \times 10^{-29} \,\mrm{m}^2.
%%\end{equation}
%%At the same time, electrons and photons are coupled via Coulomb interactions.
%As photon density and energy decreases, equilibrium is broken. Specifically when $E_\gamma < 13.6 \,\mrm{eV}$. Photons can no longer ionize hydrogen and the above relation can only go in one direction, i.e. $\longleftrightarrow \: \Rightarrow \:\longrightarrow$.
%
%Meanwhile, the universe is expanding and therefore the characteristic energy of the photons is redshifted. With time, the hydrogen dissociation rate also falls, and electrons and protons combine to form neutral hydrogen. As we learned in quantum mechanics, the ground state energy of the hydrogen atom is 13.6\,eV. However, the photon number density is significantly greater than baryon number density (see Figure \ref{fig:blackbody}).
%
%\begin{figure}[t]
%\begin{center}
%    \includegraphics*[angle=0,width=0.8\textwidth]{img/blackbody.png}
%    \caption[The blackbody spectrum]{The spectral radiance of a blackbody with temperature $T = 3000\,K$. The tail of the photon spectral radiance distribution exceeds the ground state energy of the hydrogen atom, 13.6\,eV.}
%\label{fig:blackbody}
%\end{center}
%\end{figure}
%
%\subsection{Order of Magnitude Calculations}
%\textit{This section from Rahman Amanullah 2015 lecture notes.}
%
%We can do an order of magnitude estimation to determine at what temperature and when this happens. The temperature today ($z = 0$) is $T_0 \approx 2.73$. The Stefan-Boltzmann relation tells us that the total radiation energy density today is given as
%\begin{equation}
%\rho_\gamma^0 = \sigma _0T_0^4 = 0.25 \,\mrm{eV/cm}^3
%\end{equation}
%with $\sigma _0 = 4.7 \times 10^{-3}\,\mrm{ev}/\mrm{cm}^3/\mrm{K}^4$. Similarly, we have
%\begin{equation}
%\rho _\mrm{crit}^0  = \frac{3H_0^2}{8\pi G} \approx 10^4 \,\mrm{eV}/\mrm{cm}^3,
%\end{equation}
%which suggests that $\Omega_\gamma \approx 10^{-5}$. In other words, photons only make up a tiny fraction of total energy density today. We have shown that one can write (see Chapter 4 in B\&G)
%\begin{align}
%\rho _M &= \rho_M^0(1+z)^3, \\
%\rho _\gamma &= \rho _\gamma^0 (1+z)^4.
%\end{align}
%Comparing the two equations above, we find that $\rho_M \approx \rho _\gamma$ happens when
%\begin{equation}
%(1+z_\mrm{eq}) \approx \frac{\rho_M^0}{\rho_\gamma^0} \approx 10^4.
%\end{equation}
%Here, $z_\mrm{eq}$ represents the epoch of matter-radiation equality. The recombination event, however, happens long after matter-radiation equality. To explain this, we note that the number density of photons relative to protons during recombination is very large
%\begin{equation}
%\frac{n_\gamma}{n_H} \approx \frac{\rho _\gamma/E_\gamma}{\rho _M / m_H} \approx \frac{m_H}{E_\gamma} \approx \frac{1\,\mrm{GeV}}{1\,\mrm{eV}} \approx 10^9.
%\end{equation}
%Here, $E_\gamma = (1+z_\mrm{eq})E_\gamma^0 \sim 1\,\mrm{eV}$, i.e., the temperature at $z = z_\mrm{eq}$, and the hydrogen energy at this temperature is dominated by the proton mass.
%
%Since there are more photons than baryons, recombination does not take place when the effective temperature of the universe corresponds to 13.6\,eV (the hydrogen ionization energy), but rather at 0.25\,eV. This can be understood from looking at the tail of the blackbody distribution (see Figure \ref{fig:blackbody}).
%
%For a more robust discussion of recombination, please take a look at the discussion of Saha equations in Section 9.4 in B\&G.
%
%\section{Reionization}
%The universe became electrically neutral at the time of recombination. However, we observe that the hydrogen in the interstellar medium is now largely ionized. This reionization of matter occurred at around redshift of $z \approx 10$, and is thought to have been sourced by ultraviolet radiation from the first luminous objects. We often say that the CMB photons last interacted with matter during recombination ($z = 1090$). However, about 5--10\% of the CMB photons actually last scattered on charged particles that followed reionization.
%
%The details of the reionization history of our universe are still very much a mystery! We hope to learn more about this time period as so-called 21-cm experiments (radio telescopes) advance. We can also learn something about reionization by studying the polarization of the CMB on the largest angular scales.
%
%\section{Homogeneity and Isotropy}
%\label{sec:homo}
%
%\textit{Eiichiro Komatsu's lecture notes are incredibly clear. The content of this section is influenced by the structure of his lectures. Please take a look at pages 2--5 from: \\
%\href{http://wwwmpa.mpa-garching.mpg.de/$\sim$komatsu/cmb/lecture_NG_iucaa_2011.pdf}{http://wwwmpa.mpa-garching.mpg.de/$\sim$komatsu/cmb/lecture\_NG\_iucaa\_2011.pdf}.}
%
%\begin{wrapfigure}{}{0.25\textwidth}
%\begin{center}
%\includegraphics[width=0.2\textwidth]{img/vectors.png}
%\caption[Vectors]{Vectors. \label{fig:vectors}}
%\end{center}
%
%\end{wrapfigure}
%
%Let's assume that we are interested in some location-dependent random variable, $X(\v{q})$. More specifically, we are interested the so-called 2-point correlation function (also known as the covariance matrix). The 2-point correlation function is denoted as
%\begin{equation}
%\xi _{ij} = \xi _{ij}(\v{q}_i,\v{q}_j) = \llangle X(\v{q}_i) X(\v{q}_j) \rrangle,
%\end{equation}
%it represents the likelihood of measuring a value $X(\v{q}_i)$ given a measurement at some other location $X(\v{q}_j)$. Let $\v{r}_{ij}$ be a vector connecting $\v{q}_i$ to $\v{q}_j$ (see Figure \ref{fig:vectors}), then
%\begin{equation}
%\xi _{ij} = \llangle X(\v{q}_i) X(\v{q}_i+\v{r}_{ij}) \rrangle.
%\end{equation}
%\textbf{Homogeneity} (translational invariance) means that $\xi _{ij}$ does not depend on the location $\v{q}_i$. We should therefore be able to write
%\begin{equation}
%\xi _{ij}(\v{q}_i, \v{r}_{ij}) = \llangle X(\v{q}_i) X(\v{q}_i+\v{r}_{ij}) \rrangle = \xi _{ij}(\v{r}_{ij})
%\end{equation}
%\textbf{Isotropy} (rotational invariance) means that $\xi _{ij}$ does not depend on the direction of $\v{r}_{ij}$. In other words, the property that we are studying is rotational symmetric. This means that $\xi _{ij}(\v{q}_i, \v{r}_{ij}) = \xi _{ij}(\v{q}_i, |\v{r}_{ij}| )$.
%
%By assuming homogeneity and isotropy, the 2-point correlation function therefore only depends on the magnitude of the vector connecting the two locations, $|\v{r}_{ij}|$.
%\begin{equation}
%\xi _{ij} (\v{q}_i,\v{q}_j) = \xi _{ij}(|\v{r}_{ij}|).
%\end{equation}
%As far as we can tell, we live in a homogenous and isotropic universe (at least on sufficiently large scales). As the derivation above shows, when probing the statistical properties of physical quantities, invoking homogeneity and isotropy greatly simplifies the problem. We will take another look at this in Lecture 12.
%
%\section{The Sachs-Wolfe Effect}
%
%Let's revisit Equation \ref{eq:sw}
%\begin{equation*}
%\frac{\delta T}{T}(\v{n}) = -\frac{1}{3}\Phi(\v{n}).
%\end{equation*}
%This equation can be derived by looking at the geodesic equation for photons propagating in with gravitational perturbation $\Phi$ (see Lecture 3 and Chapters 3, 4, and 11 in B\&G). We start by writing the temperature shift of a photon propagating through a gravitational well as
%\begin{equation}
%\frac{\Delta T}{T}  \bigg\rvert _f = \frac{\Delta T}{T}  \bigg\rvert _i - \Phi_i,
%\end{equation}
%where $i$ and $f$ refer to the initial and final states. The first term on the right-hand side is the ``intrinsic'' temperature at early times. The second term represents the energy that is lost when a photon climbs out of a potential well.
%
%\begin{figure}[t]
%\begin{center}
%    \includegraphics*[angle=0,width=0.8\textwidth]{img/sachs_wolfe.png}
%    \caption[The Sachs-Wolfe Effect]{An illustration of the Sachs-Wolfe Effect. A photon traversing through space encounters a gravitational well characterized by a primordial density perturbation.}
%\label{fig:sw}
%\end{center}
%\end{figure}
%
%
%Let's consider the case of adiabatic fluctuations in a matter dominated universe. Because we assume adiabaticity, we can safely assume that the intrinsic temperature fluctuations are proportional to the strength of the gravitational potential, in other words
%\begin{equation}
%\frac{\Delta T}{T} \bigg\rvert _i \sim -\Phi_i.
%\label{eq:sw1}
%\end{equation}
%
%Adiabaticity implies that matter and radiation are perturbed in similar way and that entropy per particle is constant, regardless of whether it is a photon, baryon, or some other matter. This allows us to equate temperature and density fluctuations.
%
%One could say that adiabaticity describes the thermal equilibrium between different energy constituents at the time when these ingredients were coupled. In other words, that the density contrast
%\begin{equation}
%\frac{\delta \rho _x}{\rho _x} \approx \frac{\delta \rho _y}{\rho _y}
%\end{equation}
%regardless of what energy form $x$ and $y$ represent.
%
%We want to relate initial fluctuations in the CMB temperature field to the scale factor. To achieve this, we remember that clocks run slow in a gravitational potential and that
%\begin{equation}
%ds = \sqrt{1-2\Phi}\,dt \approx (1-\Phi)dt,
%\end{equation}
%with $\Phi \ll 1$. Since the temperature of the CMB is redshifting with $aT = \mrm{constant}$ (see Section 11.1 in B\&G), we find that
%\begin{equation}
%\frac{\Delta T}{T} = -\frac{\Delta a}{a}.
%\label{eq:sw2}
%\end{equation}
%We also remind ourselves that the scale factor can be linked to time according to $a(t) = t^{\frac{2}{3(1+w)}}$ which gives
%\begin{equation}
%\delta a  = \frac{da}{dt} \delta t = \frac{2}{3(1+w)}t^{\frac{2}{3(1+w)}-1} \delta t,
%\end{equation}
%and therefore
%\begin{equation}
%\frac{\delta a}{a} = \frac{2}{3(1+w)} \frac{\delta t}{t}.
%\label{eq:sw3}
%\end{equation}
%We can now combine our results from Equations \ref{eq:sw1}, \ref{eq:sw2}, and \ref{eq:sw3} and find that
%\begin{equation}
%-\frac{2}{3(1+w}\frac{\delta t}{t} = -\frac{2}{3(1+w)}\Phi _i,
%\end{equation}
%which finally gives us
%\begin{equation}
%\frac{\Delta T}{T} \bigg\rvert_f = \frac{1+3w}{3+3w} \Phi _i.
%\end{equation}
%For a matter dominated universe we have $w = 0$ which gives us Equation \ref{eq:sw}.
%
%The Sachs-Wolfe Effect describes a contest between gravitational redshift and heating caused by local compression of matter and photons. The two effects somewhat cancel, but the net effect is that overdense regions on the sky show temperature anisotropies that are slightly cooler than average. By measuring the amplitude of the CMB temperature anisotropies, we are also constraining the amplitude of the gravitational potential variations.
%
%\begin{figure}[]
%\begin{center}
%    \includegraphics*[angle=0,width=0.5\textwidth]{img/orbit.png}
%    \caption[Orbit around sun]{Our motion through space can be decomposed into motion due to our rotation around the Sun and a component that describes the overall motion of our Solar System.}
%\label{fig:orbit}
%\end{center}
%\end{figure}
%
%\section{CMB Anisotropies}
%
%\subsection{The CMB Dipole}
%The CMB dipole is caused by our motion relative to the last scattering surface. As far as we can tell, the dipole signal can be decomposed into a component due to our motion around the sun and a component due the motion of our solar system (see Figure \ref{fig:orbit}). The amplitude of the dipole signal corresponds to about $(3355 \pm 8)\,\mrm{\mu K}$ or about 0.1\% of the amplitude of the CMB monopole. The best fit measurements suggest that our peculiar velocity is $v = 370 \,\mrm{km/s}$ in the direction $(\mrm{lat,lon}) = (264^\circ, 48.2^\circ)$.
%
%\begin{figure}[t]
%\begin{center}
%    \includegraphics*[angle=0,width=0.95\textwidth]{img/spectra_l4_v3.png}
%    \caption[The CMB temperature power spectrum]{The temperature power spectrum from a few prominent experiments. The solid line shows the best fit power spectrum derived from \planck data in conjunction with polarization data from WMAP, high-$\ell$ experiments, and experiments measuring baryon acoustic oscillation \cite{Planck2013_parameters}. \emph{Inset}: The power spectrum plotted using a different scaling on the y-axis to highlight its oscillatory nature and the available measurements. The leftmost peak at $\ell \sim 800$ is the third acoustic peak. Data obtained from the Legacy Archive for Microwave Background Analysis (LAMBDA) \cite{LAMBDA}.}
%\label{fig:tt_spectra}
%\end{center}
%\end{figure}
%
%\subsection{Primary CMB Anisotropies}
%The primary CMB anisotropies correspond to $\mathcal{O}(10 \,\mrm{\mu K})$ variations in the intensity of the CMB. Outside of the Galactic plane, these anisotropies dominate the sky signal over a wide range of angular scales. The CMB anisotropies were first mapped by the \cobe satellite and have since then be used to obtain strong constraints on cosmological models.
%
%The angular power spectrum has been measured by numerous experiments (see Figure~\ref{fig:tt_spectra}). The \planck survey covered the full sky and consequently the \planck derived $TT$ power spectrum estimate spans a remarkably wide $\ell$-range, corresponding to angular scales of 180 degrees down to approximately 3 arcmin. In Figure~\ref{fig:tt_spectra} the acoustic oscillations can be seen as a series of peaks and throughs starting at degree angular scales, coinciding with $\ell \approx 100$. Normally the power spectra are plotted as a function of $\ell$, the multipole moment. The conversion to corresponding angular scales is found by the approximate expression $\theta \approx \pi / \ell \:[\mrm{rad}]$.
%
%CMB temperature anisotropies, $T(\vhat{n}) = \delta T(\vhat{n})/T_0$, are naturally decomposed using spherical harmonics according to
%\begin{equation}
%T(\vhat{n}) = \sum _{\ell =1}^{\infty}\sum _{m=-\ell}^{\ell} a_{\ell m}^T Y_{\ell m} (\theta,\phi),
%\label{eq:deltaT}
%\end{equation}
%%where
%%\begin{equation}
%%a_{\ell m}^{T} = \int\d\Omega \:T(\theta,\phi) Y_{\ell m} (\theta,\phi)
%%\end{equation}
%%represent the expansion coefficients of a spherical harmonic decomposition. Assuming rotational invariance, the following relation must hold
%%\begin{equation}
%%\llangle a_{\ell m}^{T} a_{\ell ' m'}^{T*} \rrangle = \delta _{\ell \ell '} \delta _{m m'} C_\ell^{TT}
%%\end{equation}
%%where $\llangle \:\cdot\: \rrangle$ implies an average over the statistical ensemble defined by infinite sky realizations drawn from the same underlying theory.
%
%Since the spherical harmonics represent a complete function basis on the unit sphere, we should be able to decompose any square-integrable scalar function using the spherical harmonics. In other words, any function $T(\theta,\phi)$ can be written as
%\begin{equation}
%T(\theta,\phi) = \sum _{\ell=0}^\infty \sum_{m = -\ell}^{\ell} a_{\ell m}^T Y_\ell^m (\theta,\phi).
%\end{equation}
%If we assume that the temperature aniostropies of the cosmic microwave background (CMB) are analytic, this expansion should therefore completely capture the information in the CMB. Now, since the spherical harmonic functions are orthogonal, we also know that
%\begin{equation}
%a_{\ell m}^T = \int d\Omega T(\theta,\phi)Y_\ell^m(\theta,\phi)
%\end{equation}
%where the $d\Omega = \sin \theta d\theta d\phi$ integral is performed over the full area of the unit sphere. The spherical harmonic expansion of a scalar function is therefore calculated by performing an integral on the unit sphere for each function used in the expansion. We also know that since these functions are orthonormal (both orthogonal and normalized)
%\begin{equation}
%\int d\Omega Y_\ell^m (\theta,\phi) Y_{\ell'}^{m'} (\theta,\phi) = \delta _{\ell \ell'} \delta_{m m'}
%\end{equation}
%
%Let's define the $\left< \cdot \right>$ operator as the ensemble average defined by infinite sky realizations drawn from the same underlying theory and use this to define the correlation between spherical harmonic coefficients
%\begin{equation}
%C_{\ell m \ell' m'}^{TT} \equiv \:  \left< a_{\ell m}^T \: a_{\ell ' m'}^{T*} \right>.
%\end{equation}
%If the CMB anisotropies are homogeneous and isotropic, in other words, rotationally invariant, one can argue that
%\begin{equation}
%C_{\ell m \ell' m'}^{TT} = C_\ell^{TT} \delta _{\ell \ell'} \delta _{m m'}.
%\end{equation}
%This gives
%\begin{equation}
%C_{\ell}^{TT}\delta _{\ell \ell'} \delta _{m m'} = \:  \left< a_{\ell m}^T \: a_{\ell ' m'}^{T*} \right>.
%\end{equation}
%
%So what happens when we try to find the expectation value of $T(\theta,\phi)$ over the sky? Well, we've already removed the CMB monopole from our maps (remember we're talking about the anisotropies so $\left< T(\theta,\phi) \right> = 0$. What about the temperature anisotropy covariance?
%\begin{align}
%\left<T(\hat{n}) T(\hat{n})'\right> &= \left< \sum _{\ell=0}^{\infty} \sum _{m=-\ell}^{\ell} a_{\ell m}^T Y_{\ell m} (\theta, \phi) \sum _{\ell'=0}^{\infty} \sum _{m'=-\ell'}^{\ell'} a_{\ell' m'}^T Y_{\ell' m'} (\theta ', \phi ') \right>, \\
%&= \left< \sum _{\ell=0}^{\infty} \sum _{m=-\ell}^{\ell} |a_{\ell m}^{T}|^2 Y_{\ell m} (\theta, \phi) Y_{\ell m} (\theta ', \phi ') \right>.
%\end{align}
%To arrive at this expression we've assume rotational invariance to collapse the $\ell '$ and $m '$ moments and simplify our expression. It follows that
%\begin{equation}
%\llangle T(\vhat{n}) T(\vhat{n}') \rrangle = \frac{1}{4\pi}\sum _{\ell =1}^{\infty} (2\ell+1) C_\ell^{TT} P_{\ell} (\vhat{n} \cdot \vhat{n}'),
%\label{eq:CTT}
%\end{equation}
%where $P_{\ell} (\vhat{n} \cdot \vhat{n}')$ is a Legendre polynomial of order $\ell$ and the following mathematical identity has been used (see Figure \ref{fig:legendre}):
%\begin{equation}
%P_{\ell} (\vhat{n} \cdot \vhat{n}') = \frac{4\pi}{2\ell+1} \sum _{m=-\ell}^{\ell} Y_{\ell m}(\vhat{n}) Y_{\ell m} (\vhat{n}').
%\end{equation}
%The angular power spectrum, $C_\ell^{TT}$, then represents the variance in power in a given $\ell-$mode. In other words, the correlation between two points on the sphere can be represented by a series formed by the product of multipole-space power spectra and Legendre polynomials.
%
%\begin{figure}[]
%\begin{center}
%    \includegraphics*[angle=0,width=0.9\textwidth]{img/legendre.png}
%    \caption[Legendre Polynomials]{First 10 Legendre Polynomials.}
%\label{fig:legendre}
%\end{center}
%\end{figure}
%
%%The next step in simplifying this expression involves invoking the the \textbf{addition theorem for Legendre functions}. I feel that many cosmology text books jump over this crucial part of the derivation. The addition theorem states that
%%\begin{equation}
%%P_\ell (\hat{n} \cdot \hat{n}') = \frac{4\pi}{2\ell +1} \sum_{m=-\ell}^\ell Y _{\ell m} (\hat{n}) Y_{\ell m} (\hat{n}')
%%\end{equation}
%%where $\hat{n}$ and $\hat{n}'$ are two unit vectors (directions on the sky). Continuing from where we left off, this gives us
%%\begin{equation}
%%\left<T(\hat{n}) T(\hat{n})'\right> = \frac{1}{4\pi} \sum_{\ell=0}^\infty C_\ell (2\ell + 1) P_\ell (\cos \theta).
%%\end{equation}
%%We often write $C(\theta) = \left<T(\hat{n}) T(\hat{n})'\right>$ and call this the angular correlation function. The above expression is of fundamental importance for understanding the angular power spectrum of the cosmic microwave background.
%
%From all of this we have learned that the RMS temperature anisotropies associated with a particular multipole can be written as
%\begin{equation}
%\Delta T_\ell = \sqrt{ \frac{\ell (\ell +1) \:C_\ell}{2\pi}}.
%\end{equation}
%
%\begin{multicols}{2}
%\linespread{0.5}
%\setlength{\bibsep}{0pt plus 0.3ex}
%\begingroup
%%\bibliographystyle{unsrt85}
%\bibliographystyle{unsrtnat}
%%\bibliographystyle{science}
%%\setlength{\bibitemsep}{-5pt}
%\linespread{0.5}\selectfont
%{\scriptsize \bibliography{lecture12}}
%\endgroup
%\end{multicols}
%
%\section*{Appendix A}
%The spherical harmonics represent an orthogonal function basis on the sphere. This means that any analytic scalar function, $\tau(\theta,\phi)$, can be described by the series:
%\begin{equation}
%\tau (\theta, \phi) = \sum _{\ell,m} a_{\ell m} Y_\ell ^m (\theta,\phi)
%\end{equation}
%What do the $Y_\ell^m (\theta,\phi)$ functions look like? Before we answer that question, let's see where these functions originate. The spherical harmonic functions are the solutions to Laplace's equation on the sphere.
%\begin{equation}
%\nabla ^2 f = 0
%\end{equation}
%\textbf{Harmonic functions} are twice continuously differentiable functions that satisfy Laplace's equation. The etymology of the word harmonic refers to the motion of a string fixed at both ends. The solution to the differential equation for the type of motion that the string undergoes can be written as a series of sine and cosine functions.
%
%In spherical coordinates this equation becomes
%\begin{equation}
%\nabla ^2 f = \frac{1}{r^2} \frac{\partial}{\partial r} \left( r^2\frac{\partial f}{\partial r} \right) + \frac{1}{r^2 \sin \theta} \frac{\partial}{\partial \theta} \left( \sin \theta \frac{\partial f}{\partial \theta} \right) + \frac{1}{r^2 \sin ^2 \theta} \frac{\partial ^2 f}{\partial \phi ^2} = 0
%\end{equation}
%Remember that all harmonic functions are analytic. This means that the functions can be locally represented by a convergent power series which in turn means that the harmonic functions are infinitely differentiable.
%
%As so often in PDE's we explore solutions that allow for separation of variables. The function that describes the $\theta$ and $\phi$ dependence of the solution (imagine we've fixed $r$) is
%\begin{equation}
%Y _\ell^m (\theta, \phi) = (-1)^m \sqrt{ \frac{(2\ell +1)}{4\pi}\frac{(\ell-m)!}{(\ell+m)!}} e^{i m \phi} P_\ell^m (\cos \theta)
%\end{equation}
%where we've picked a normalization factor that makes $Y_\ell^m$ orthonormal. The associated Legendre polynomials, $P_\ell^m$, are the canonical solutions of the general Legendre equation
%\begin{equation}
%(1-x^2) \frac{d^2}{dx^2}P_\ell^m (x) -2x\frac{d}{dx} P_\ell^m (x) + \left[ \ell(\ell+1)-\frac{m^2}{1-x^2} \right] P_\ell^m (x)
%\end{equation}
%From the above equation, we see that the spherical harmonics functions are inextricably bound to the associated Legendre polynomials, $P_\ell^m$. %Note, to get a rough feel for the relation between an angular scale, $\alpha$, and  multipole moment, $\ell$, the following relation works well $\alpha = \pi / \ell$.
%
%%Let's take a look at the associated Legendre polynomials.

\begingroup
%\bibliographystyle{unsrt85}
%\bibliographystyle{unsrtnat}
\bibliographystyle{unsrtnat}
%\bibliographystyle{science}
%\setlength{\bibitemsep}{-5pt}
\linespread{0.5}\selectfont
\bibliography{lecture12}
%{\bibliography{snsb2019}}
\endgroup

\end{document}
%%
%% EOF
